\section{Completeness}

\begin{table*}
  \hrulefill\\

  % \begin{tabular}{l}
    \begin{minipage}{300pt}%
      $\forall \trace \in \Actfin, \forall \aa \in \Names,$
      \begin{enumerate}[(1)]
      \item
        \label{gen-spec-ungood}
        $ \lnot \good{f(\trace)}$
      \item
        \label{gen-spec-mu-lts-co}
        $ \forall \mu \in \Act, f (\mu.\trace) \st{\co{\mu}} f(\trace)$
      \item
        \label{gen-spec-mu-out-ex-tau}
        $ f (\co{a}.\trace) \st{\tau} $
      \item
        \label{gen-spec-out-good}
        $ \forall \client \in \States, f (\co{a}.\trace)
        \st{\tau} \client$ implies $\good{ \client }$
      \item
        \label{gen-spec-out-mu-inp}
        $ \forall \client \in \States, \mu \in \Act,$
        $f (\co{a}.\trace) \st{\mu} \client$ implies $\mu = \aa$ and
        $\client = f(s)$
      \end{enumerate}
    \end{minipage}
    \\[2em]
    \begin{minipage}{300pt}
      $\forall E \subseteq \Names$,
      \begin{enumerate}[(t1)]
      \item\label{gen-spec-acc-nil-stable-tau}
        $\testacc{\varepsilon}{E} \Nst{\tau}$
      \item\label{gen-spec-acc-nil-stable-out}
        $\forall \aa \in \Names, \testacc{\varepsilon}{E} \Nst{\co{\aa}}$
      \item\label{gen-spec-acc-nil-mem-lts-inp}
        $\forall \aa \in \Names, \testacc{\varepsilon}{E} \st{\aa}$ if and only if
        $\aa \in E$
      \item\label{gen-spec-acc-nil-lts-inp-good}
        $\forall \mu \in \Act, \client \in \States,
        \testacc{\varepsilon}{E} \st{\mu} \client$ implies $\good{\client}$
      \end{enumerate}
    \end{minipage}
  % \end{tabular}
  \\[2em]
   %%%%%%%%
  %%%%%%%% HORIZONTAL LAYOUT BY GIO
 %%%%%%%%
%% %% \multicolumn{3}{l}{%
%%   \begin{enumerate}[(c1)]
%% \item
%%   $\forall \mu \in \Act, \testconv{\varepsilon} \Nst{\mu}$ \qquad\qquad
%% \item
%%   $\testconv{\varepsilon} \st{\tau} $ \qquad\qquad
%% \item
%%   $\forall \client, \testconv{\varepsilon} \st{\tau} \client$ implies
%%   $\good{ \client }$
%%   \end{enumerate}
%%   \\[7pt]
%% %%  }
  %%%%%%%%
  %%%%%%%% VERTICAL LAYOUT BY ILARIA
  %%%%%%%%
    % \begin{tabular}{c}
      \begin{minipage}{300pt}
      \begin{enumerate}[(c1)]
      \item
        $\forall \mu \in \Act, \testconv{\varepsilon} \Nst{\tau}$ \qquad
      \item
        $\exists \client, \testconv{\varepsilon} \st{\tau} \client$ \qquad
      \item
        $\forall \client, \testconv{\varepsilon} \st{\tau} \client$ implies
        $\good{ \client }$
        \\
      \end{enumerate}
      \end{minipage}
    % \end{tabular}
  %%%%%%%%
%%%%%%%% VERTICAL LAYOUT BY ILARIA
%%%%%%%%
  \caption{Properties of the functions that generate clients.}
\hrulefill
\label{tab:properties-functions-to-generate-clients}
\end{table*}


\label{sec:bhv-completeness}
This section outlines the proof that the alternative preorder
given in \rdef{accset-leq} is included in the \mustpreorder.
Proofs of completeness of this kind usually require using syntactic
contexts. Our calculus-independent setting, though, does not allow us to
define them.  Instead we phrase our arguments using two functions
$
\testconvSym :  \Actfin \rightarrow \States,$ and
$\testaccSym :  \Actfin \times \parts{\Names} \rightarrow \States$
where \lts{ \States }{L}{ \st{} } is some LTS of \obaFB.
In \rtab{properties-functions-to-generate-clients} we gather all the
{\em properties} of~$\testconvSym$ and~$\testaccSym$ that are sufficient to give our
arguments. The properties (1) - (5) must hold for both~$\testconvSym$ and
$\testacc{\varepsilon}{-}$ for every set of names~$O$, the properties (c1) -
(c2) must hold for~$\testconvSym$, and (t1) - (t4) must hold for~$\testaccSym$.

We use the function~$\testconvSym$ to test the convergence of servers, and the
function~$\testaccSym$ to test the acceptance sets of servers.

A natural question is whether such~$\testconvSym$ and~$\testaccSym$ can actually exist.
The answer depends on the LTS at hand. In \rapp{client-generators},
and in particular \rfig{client-generators}, we define these functions for
the standard LTS of~\ACCS, and it should be obvious how to adapt those
definitions to the asynchronous $\pi$-calculus \cite{DBLP:journals/jlp/Hennessy05}.
%\pl{Two times obvious in this par. Maybe we could simply link
%  papers that define test generators for the $\pi$-calculus}.

In short, our proofs show that~$\asleq$ is complete wrt~$\testleqS$
in any LTS of output-buffered agents with feedback wherein the
functions~$\testconvSym$ and~$\testaccSym$ enjoying the properties in
\rtab{properties-functions-to-generate-clients} can be defined.


%%Instead, we define sets of properties the test generators must obey to.
%% \begin{definition}
%% We first define the properties that must be respected by the test generators
%% we rely on in this section.

%% Given a test generator $f : \Actfin \rightarrow \States$,
%% we write $\mathcal{M} \Vdash f(\trace}$ if


%% We now define the properties that must be respected
%% by the test generator used to test the convergence of
%% a server along a trace.

%% Given a test generator $c : \Actfin \rightarrow \States$,
%% we write $\mathcal{C} \Vdash \testconv{\trace}$ if
%% $\mathcal{M} \Vdash \testconv{\trace}$ and $c$ obey to these properties.

%% We now define the properties that must be respected by the test generator
%% used to test the acceptance-sets of a server.

%% Given a test generator $a : \set{\Names} \times \Actfin \rightarrow \States$,
%% we write $\mathcal{A} \Vdash \testacc{O}{\trace}$ if for any $O \subseteq \Names$ and $s \in \Actfin$,
%% $\mathcal{M} \Vdash \testacc{O}{\trace}$ and $a$ respects the following properties.



%% For the sake of clarity, we first provide the reader                     %%
%% with two test generators that obey the axioms and can be                 %%
%% are usually defined in the litterature                                   %%
%% when proving completeness in the context of ACCS. \rfig{test-generators} %%
%% \begin{example}                                                          %%
%% \testconv{\trace} = ...                                                               %%
%% a(s,O) = ...                                                             %%
%% \end{example}                                                            %%
%%                                                                          %%
%% \begin{lemma}                                                            %%
%% The two test generators obey to the axioms                               %%
%% \end{lemma}                                                              %%


%% This section is then parameterized by two test generators
%% $\testconv{\trace}$ and $a(O,\trace}$ such that $\mathcal{C} \Vdash \testconv{\trace}$
%% and $\mathcal{A} \Vdash a(s, O)$ hold.


\renewcommand{\States}{\ensuremath{A}}


First, converging along a finite trace~$\trace$~is logically
equivalent to passing the client~$\testconv{ \trace}$.  In other
words, there exists a bijection between the proofs (i.e. finite
derivation trees of~$\musti{ \server }{\testconv{ \trace}}$) and
the ones of~$ \server \cnvalong{ \trace }$. We first give the
proposition, and then discuss the auxiliary lemmas to prove it.

\begin{proposition}\coqCom{must_iff_acnv}%{myproposition}{mustiffacnv}
  \label{prop:must-iff-acnv}
  For every $\genlts_{\States} \in \obaFW$,
  $\server \in \States$, and
  $\trace \in \Actfin$ we have that $\musti{\server}{ \testconv{ \trace} }$
  if and only if~$\server \cnvalong \trace$.
\end{proposition}
\noindent
%\begin{proof}
The {\em if} implication is \rlem{acnv-must} and the {\em only if}
implication is \rlem{must-cnv}.
  %\end{proof}
\noindent
The hypothesis that $\genlts_{\States} \in \obaFW$,
\ie the use of forwarders, is necessary to show that convergence
implies passing a client, as shown by the next example.
\begin{example}
    \label{ex:forwarders-necessary}
    Consider a server~$\server$ in an LTS $\genlts \in \obaFB$
%    with, in the set $\StatesA$,
%    a state~$\server$, % that can input on~$\ab$ and then diverges.
    whose behaviour amounts to the following transitions:
    $\server \st{ \ab } \Omega \st{ \tau }  \Omega \st{ \tau } \ldots$
    Note that this entails that $\genlts$ does not
    {\em not} enjoy the axioms of forwarders.

    Now let $ \trace = \aa.\ab $. Since $\server \conv$ and $\server
    \Nwt{\aa}$ we know that $ \server \cnvalong \aa.\ab$.
    On the other hand \rtab{properties-functions-to-generate-clients}(\ref{gen-spec-mu-lts-co})
    implies that the client $\testconv{\trace}$ performs the transitions
    $\testconv{ \trace } \st{ \co{\aa}} \testconv{ \ab }  \st{ \co{\ab}} \testconv{ \varepsilon }$.
    Thanks to the \outputcommutativity axiom we obtain $\testconv{ \trace } \st{ \co{\ab}} \st{ \co{\aa}} \testconv{ \varepsilon }$.
    \rtab{properties-functions-to-generate-clients}(\ref{gen-spec-ungood}) implies that the states
    reached by the client are unsuccessful, and so by zipping the traces performed
    by $\server$ and by $\testconv{\trace}$
    we build a maximal computation of
    $\csys{\server}{\testconv{\trace}}$ that is unsuccessful,
    and thus $\Nmusti{\server}{\testconv{\trace}}$.\hfill$\qed$
  \end{example}
  \noindent
  This example explains why in spite of \rlem{musti-obafb-iff-musti-obafw}
  output-buffered agents with feedback do not suffice to use the
  standard characterisations of the \mustpreorder.


%%%%%%%%%%%%%%%%%%%%%%%%%%%%%%%%%%%%%%%%%%%%%%%%%%%%%%%%%%%%%%%%
%%%%%%%%%%%%%%%%%%%%%%%%%%%%%%%%%%%%%%%%%%%%%%%%%%%%%%%%%%%%%%%%
%%%%%%%%%%%%%%%%%%%%%%%%%%%%%%%%%%%%%%%%%%%%%%%%%%%%%%%%%%%%%%%%
%% ARGUMENTS ABOUT ACCEPTANCE SETS
%%%%%%%%%%%%%%%%%%%%%%%%%%%%%%%%%%%%%%%%%%%%%%%%%%%%%%%%%%%%%%%%
%%%%%%%%%%%%%%%%%%%%%%%%%%%%%%%%%%%%%%%%%%%%%%%%%%%%%%%%%%%%%%%%
%%%%%%%%%%%%%%%%%%%%%%%%%%%%%%%%%%%%%%%%%%%%%%%%%%%%%%%%%%%%%%%%

We move on to the more involved technical results, \ie the next
  three lemmas, that we use to reason on acceptance sets of servers.
  The proofs are deferred to \rapp{bar-induction} and \rapp{appendix-completeness},
  but we wish to stress \rlem{must-output-swap-l-fw}.
  It states that, when reasoning on~$\opMusti$,
  outputs can be freely moved from the client to the server side of
  systems, if servers %are forwarders.
  have the forwarding ability.
  Its proof uses {\em all} the axioms for output-buffered agents with
  feedback, and the lemma itself is used in the proof of the main
  result on acceptance sets, namely \rlem{completeness-part-2.2-auxiliary}.

\begin{lemma}[ Output swap ]
  \label{lem:must-output-swap-l-fw}
  Let $\genlts_A \in \obaFW$ and
  $\genlts_B \in \obaFB$.
  $\Forevery \serverA_1, \serverA_2 \in \StatesA$,
  every $\client_1, \client_2 \in \StatesB$ and name $\aa \in \Names$ such that
  $\serverA_1 \st{\co{\aa}} \serverA_2$ and
  $\client_1 \st{\co{\aa}} \client_2$,
  if $\musti{\serverA_1}{\client_2}$ then $\musti{\serverA_2}{\client_1}$.
\end{lemma}

\begin{lemma}
  \label{lem:completeness-part-2.2-diff-outputs}
  Let $\genlts_A \in \obaFW$.
  For every $\server \in \States$, $\trace \in \Actfin$,
  and every $L, E \subseteq \Names$, if
  $\co{L} \in \accht{ \server }{ \trace }$
  then $\Nmusti{ \server }{ \testacc{\trace}{E \setminus L}}$.
\end{lemma}



\begin{lemma}\coqCom{must_gen_a_with_s}
  \label{lem:completeness-part-2.2-auxiliary}
  Let $\genlts_A \in \obaFW$.
  $\Forevery \server \in \States, \trace \in \Actfin$,
  and every finite set $\ohmy \subseteq \co{\Names}$,
  if $\server \cnvalong s$ then either
  \begin{enumerate}[(i)]
      \item
    %% \item\label{pt:completeness-crux-move-1} %%
    $\musti{\server}{\testacc{ \trace }{ \bigcup \co{ \accht{p}{s}
          \setminus \ohmy }}}$, or
  \item
   %% \item\label{pt:completeness-crux-move-2} %%
    there exists $\widehat{\ohmy} \in \accht{ \server }{ \trace }$ such that $\widehat{\ohmy} \subseteq \ohmy$.
  \end{enumerate}
\end{lemma}

\renewcommand{\traceA}{s_1}
\renewcommand{\traceB}{s_2}

%% \begin{lemma}%[\coqConv{acnv_weak_a_congr}]%{mylemma}{acnvsplits}                 %%
%%  \label{lem:acnv-split-s}                                                         %%
%%  For every $\traceA, \traceB \in \Actfin,$ and                                    %%
%%  $\serverA, \serverB  \in \States$,                                               %%
%%  if $\serverA \cnvalong \traceA.\traceB$ and $ \serverA \wta{ \traceA} \serverB$ %%
%% then $\serverB \cnvalong \traceB$.                                               %%
%% % $ (p \aftera \traceA)  \cnvalong \traceB$.                                     %%
%% \end{lemma}                                                                       %%

\renewcommand{\stateB}{q}
\renewcommand{\traceB}{\traceC}

\renewcommand{\traceA}{s_1}
\renewcommand{\traceB}{s_2}
\renewcommand{\traceC}{s_3}

%% Now we gather the properties of the clients generated by the function~$g$~and~$c$. %%
%% \begin{lemma}%[\coqCom{gen_test_reduces_if}]                                        %%
%%   \label{lem:gen-reduces-if}                                                        %%
%%   \label{lem:gs-reduces}                                                            %%
%%   For every $\server \in \Proc$ and                                                 %%
%%   $\trace \in \Actfin$, if $p \st{\tau}$ then $g(\trace, \server) \st{\tau}$.       %%
%% \end{lemma}                                                                         %%
%%%% ORIGINAL ARGUMENT
%%%% DO NOT EREASE
%% \begin{proof}
%% The proof is by induction on the sequence $s$.
%% In the base case $s = \varepsilon$. The test generated by $g$ is $p$,
%% which reduces by hypothesis, and so does $g(\varepsilon,p)$.
%% In the inductive case $s = \mu.s'$.
%% We proceed by case-analysis on $\mu$.
%% If $\mu$ is an output then $g(\mu.s', p) = \co{\mu}.(g(s', p)) \extc
%% \tau.\Unit$ which reduces to $\Unit$ using
%% the transition rule \extR.
%% If $\mu$ is an input then $g(\mu.s', p) = \co{\mu} \Par g(s', p)$
%% which reduces using the transition rule
%% \parR and the inductive hypothesis, which ensures that
%% $g(s', p)$ reduces.
%% \leaveout{%
%% and the hypothesis tells us that it reduces.
%% In the inductive case $s = \mu.s'$ and our inductive hypothesis tells
%% us that $g(s', p)$ reduces.
%% We proceed by case-analysis on $\mu$.
%% If $\mu$ is an input then $g(\mu.s', p) = \co{\mu} \Par g(s', p)$
%% which reduces using the transition rule
%% \parR and the inductive hypothesis.
%% If $\mu$ is an output then $g(\mu.s', p) = \co{\mu}.(g(s', p)) \extc
%% \tau.\Unit$ which reduces to $\Unit$ using
%% the transition rule \extR.}
%% \end{proof}
%%%% ORIGINAL ARGUMENT
%%%% DO NOT EREASE
%%%%%%%%%%%%%%%%%%%%%%%%%%%%%%%%%%%%%%%%%%%%%%%%%%%%%%%%%


\renewcommand{\traceA}{s_1}
\renewcommand{\traceB}{s_2}
\renewcommand{\traceC}{s_3}





%% \begin{lemma}%[\coqCom{must_gen_conv_wta_mu}]                                               %%
%%   \label{lem:must-weak-a-mu}                                                                %%
%%   For every $\mu \in \Act, \trace \in \Actfin$ and                                          %%
%%   $\server \in \Proc \wehavethat  \musti{\server}{\testconv{\mu.\trace)}$                           %%
%%   and $\server \wt{\mu} \server'$ imply that %$\musti{ (\server \aftera \mu) }{\testconv{\trace)}$. %%
%%   $\musti{ \server' }{\testconv{\trace)}$.                                                          %%
%% \end{lemma}                                                                                 %%
%%%%%%%%%%%%%%%%%%%%%%%%%%%%%%%%%%%%%%%%%%%%%%%%%
%%%%%%%%%%%%%%%%%%%%%%%%%%%%%%%%%%%%%%%%%%%%%%%%%
%%%%%%%% ORIGINAL ARGUMENT
%%%%%%%% DO NOT DELETE
%% \begin{proof}
%%   We have to show that $q \musti \testconv{\trace}$ for every $q$ such that $p \wta{\mu} q$.
%%   Pick such $q$, and decompose the weak transition $p \wta{\mu} q$ as
%%   follows,
%%   $$
%%   p \wta{\varepsilon} \widehat{p} \sta{\mu} \widehat{q} \wta{\varepsilon} q
%%   $$

%%   We proceed by induction on the derivation of $p \wta{\varepsilon} \widehat{p}$.

%%   In the base case $p = \widehat{p}$.
%%   Since $\testconv{ \mu.s ) \Nst{\ok}$ and  $\testconv{\mu.\trace} \st{\co{\mu}} \testconv{\trace}$\footnote{This fact follows directly from the definition of $\testconv{\trace}$, and formally we use \rlem{inversion-gen-mu}},
%%   \rlem{musti-presereved-by-actions-of-unsuccesful-tests}
%%   ensures that $\widehat{q} \musti \testconv{\trace}$.
%%   \rlem{must-lts-a-tau} together with an induction on
%%   the derivation of $\widehat{q} \wta{\varepsilon} q$
%%   ensure that $\tmusti{q}{\testconv{\trace}}$ as required.


%%   In the inductive case there exists $p'$ such that
%%   $p \sta{\tau} p' \wta{\varepsilon} \widehat{p}$ and that
%%   $$
%%   p \sta{\tau} p' \wta{\varepsilon} \widehat{p} \sta{\mu} \widehat{q} \wta{\varepsilon} q
%%   $$
%%   The inductive hypothesis esures that
%%   $\tmusti{p'}{\testconv{\mu.\trace}}$ implies $\tmusti{q}{\testconv{\trace}}$,
%%   and thus we obtain $\tmusti{q}{\testconv{\trace}}$ via an application of
%%   \rlem{must-lts-a-tau}.
%% \end{proof}
%%%%%%%% DO NOT DELETE
%%%%%%%%%%%%%%%%%%%%%%%%%%%%%%%%%%%%%%%%%%%%%%%%%
%%%%%%%%%%%%%%%%%%%%%%%%%%%%%%%%%%%%%%%%%%%%%%%%%





%% First we give sufficient conditions for a server                                                                         %%
%% not to satisfy a client generated by the function $\testacc{-}{-}$, then                                                    %%
%% we discuss \rlem{completeness-part-2.2-diff-outputs}. The other proofs                                                   %%
%% are in \rapp{appendix}.                                                                                                  %%
%%                                                                                                                          %%
%%                                                                                                                          %%
%% \begin{lemma}                                                                                                            %%
%%   \label{lem:gen-test-unhappy-if}                                                                                        %%
%%     \label{lem:gen-test-lts-co-mu}                                                                                       %%
%%   For every $\client \in \Proc$ and $\trace \in \Actfin$,                                                                %%
%%   \begin{enumerate}                                                                                                      %%
%%   \item %(\coqCom{gen_test_unhappy_if})                                                                                  %%
%%     \label{pt:gen-test-unhappy-if}   if $\lnot \good{\client}$ then                                                      %%
%%   $\lnot \good{g( \trace , \client )}$,                                                                                  %%
%%   \item %(\coqCom{gen_test_lts_mu})                                                                                      %%
%%     \label{pt:gen-test-lts-co-mu} for every $\mu \in \Act$,  $g(\mu.\trace, \client) \st{\co{\mu}} g(\trace , \client)$. %%
%%   \end{enumerate}                                                                                                        %%
%% \end{lemma}                                                                                                              %%


We can now show that the alternative preorder~$\asleq$
includes~$\testleqS$ when used over LTSs of forwarders.
\begin{lemma}%, \coqCom{completeness}]
  \label{lem:completeness}
  For every $\genlts_A, \genlts_B \in \obaFW$ and
  servers $\serverA \in \StatesA, \serverB \in \StatesB $,
  if ${ \serverA } \testleqS { \serverB }$
  then ${ \serverA } \asleq { \serverB }$.
\end{lemma}
\begin{proof}
  Let ${ \serverA } \testleqS { \serverB }$.
  To prove that ${ \serverA } \bhvleqone { \serverB }$,
  suppose
  ${\serverA} \cnvalong \trace$ for some trace $\trace$.
  \rprop{must-iff-acnv} implies $\musti{ {\serverA} }{\testconv{ \trace} }$,
  and so by hypothesis $ \musti{ { \serverB } }{\testconv{ \trace }}$.
  \rprop{must-iff-acnv} ensures that ${\serverB} \cnvalong \trace$.

  We now show that ${ \serverA } \bhvleqtwo { \serverB }$.
  Thanks to
  \rlem{conditions-on-accsets-logically-equivalent}, it is enough to prove
  that $\serverA \asleqAfw \serverB$. So, we show that
  for every trace $\trace \in \Actfin$, if ${ \serverA } \acnvalong \trace$
  then $\accht{{ \serverA }}{\trace} \ll
  \accht{{ \serverB }}{\trace} $.
  Fix an $O \in \accht{ {\serverB} }{ \trace }$.
  We have to exhibit a set $\widehat{O} \in \accht{{ \serverA
  }}{\trace}$ such that $\widehat{O} \subseteq O$.


  By definition $O \in \accht{ {\serverB} }{ \trace }$
  means that for some $q'$ we have ${\serverB} \wt{ \trace } q' \stable$
  and $O(\serverB') = O$.
  Let $E = \bigcup \accht{ { \serverA }}{\trace}$ and $X = E \setminus O $.
  The hypothesis that ${\serverA}~\cnvalong~\trace$,
  and the construction of the set~$X$
  let us apply \rlem{completeness-part-2.2-auxiliary}, which implies that
  either \begin{enumerate}[(a)] \item
  $\musti{{\serverA}}{\testacc{\trace}{ \co{ X }}}$, or
  \item there exists
  a $\widehat{O} \in \accht{{ \serverA }}{\trace}$ such that $\widehat{O}
  \subseteq O(\serverB')$.
  \end{enumerate}
  Since (b) is exactly what we are after, to conclude the
  argument it suffices to prove that (a) is false.
  This follows from \rlem{completeness-part-2.2-diff-outputs}, which proves
  $\Nmusti{ {\serverB}  }{ \testacc{ \trace }{ \co{ X }} }$,
  and the hypothesis ${ \serverA } \testleqS { \serverB }$,
  which ensures  $\Nmusti{ {\serverA}  }{ \testacc{ \trace }{ \co{ X }} }$.
\end{proof}

The fact that the \mustpreorder can be captured via the function $\liftFW{-}$
and $\asleq$ is a direct consequence of \rlem{musti-obafb-iff-musti-obafw}
and \rlem{completeness}.
\begin{proposition}[Completeness]%, \coqCom{completeness}]
  \label{prop:bhv-completeness}
  For every $\genlts_A, \genlts_B \in \obaFB$ and
  servers $\serverA \in \StatesA, \serverB \in \StatesB $,
  if $\serverA \testleqS \serverB$ then $\liftFW{ \serverA } \asleq \liftFW{ \serverB }$.
\end{proposition}
%%%%%
  %%%%%%%%%%%%%%%%%%%%%%%%% OLD PROOF FOR LICS. DO NOT EREASE BEFORE DEADLINE
  %%%%%%%%%%%%%%%%%%%%%%%%%
%%\begin{proof}
  %% We have to prove the following set inclusions
  %% \begin{align}
  %%   {\testleqS} & \subseteq {\bhvleqone} \tag{cmp-1 \coqCom{completeness1}}\\
  %%   {\testleqS} & \subseteq {\bhvleqtwo} \tag{cmp-2 \coqCom{completeness2}}
  %% \end{align}
  %% To prove the first inclusion, suppose
  %% $\serverA \testleqS \serverB$ and $\liftFW{\serverA} \cnvalong \trace$ for some trace $\trace$.
  %% \rprop{must-iff-acnv} implies
  %% $\musti{ \liftFW{\serverA} }{\testconv{ \trace}}$.
  %% \rlem{musti-obafb-iff-musti-obafw} implies that $ \musti{ \serverA }{\testconv{ \trace }} $,
  %% thus the hypothesis implies that $\musti{ \serverB }{\testconv{ \trace }}$.
  %% Applying again \rlem{musti-obafb-iff-musti-obafw} and
  %% \rprop{must-iff-acnv} we obtain $\liftFW{ \serverB } \cnvalong \trace$.
  %%
  %% To prove the second inclusion, thanks to
  %% \rlem{conditions-on-accsets-logically-equivalent} we have to explain
  %% why for every trace $\trace \in \Actfin$ if $\liftFW{ \serverA } \acnvalong \trace$
  %% then it is the case that $\accht{\liftFW{ \serverA }}{\trace} \ll
  %% \accht{\liftFW{ \serverB }}{\trace} $.
  %% Fix an $O \in \accht{ \liftFW{\serverB} }{ \trace }$.
  %% We have to exhibit a set $\widehat{O} \in \accht{\liftFW{ \serverA
  %% }}{\trace}$ such that $\widehat{O} \subseteq O$.
  %%
  %% By definition $O \in \accht{ \liftFW{\serverB} }{ \trace }$
  %% means that for some $q'$ we have $\liftFW{\serverB} \wt{ \trace } q' \stable$
  %% and $O(\serverB') = O$.
  %% Let
  %% %$ \mathcal{O}_{\serverA} =  \bigcup \setof{ O }{ O \in \accht{ \serverA }{ \trace } }$
  %% $E = \bigcup \accht{ \liftFW{ \serverA }}{\trace}$
  %% and $X = E \setminus O $.
  %% The hypothesis of convergence, \ie that $\liftFW{\serverA}~\cnvalong~\trace$,
  %% and the construction of the set~$X$
  %% let us apply \rlem{completeness-part-2.2-auxiliary} which implies that
  %% either \begin{enumerate}[(a)] \item
  %% $\musti{\liftFW{\serverA}}{\testacc{\trace}{ \co{ X }}}$, or
  %% \item there exists
  %% a $\widehat{O} \in \accht{\liftFW{ \serverA }}{\trace}$ such that $\widehat{O}
  %% \subseteq O(\serverB')$.
  %% \end{enumerate}
  %% Since (b) is exactly what we are after, to conclude the
  %% argument it suffices to prove that (a) is false. This is done in
  %% four steps:
  %% we apply \rlem{completeness-part-2.2-diff-outputs} to prove
  %% $\Nmusti{ \liftFW{\serverB}  }{ \testacc{ \trace }{ \co{ X }} }$,
  %% then we apply
  %% \rlem{musti-obafb-iff-musti-obafw} to obtain
  %% $\Nmusti{\serverB}{ \testacc{ \trace }{ \co{ X }} }$. This fact together
  %% with $\serverA \testleqS \serverB$ imply
  %% $\Nmusti{ \serverA }{ \testacc{ \trace }{ \co{ X }} }$.
  %% Applying again \rlem{musti-obafb-iff-musti-obafw} we conclude
  %% $\Nmusti{ \liftFW{\serverA} }{ \testacc{ \trace }{ \co{ X }} }$.
%%%\end{proof}
