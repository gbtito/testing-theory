\newpage
\section{Forwarders}
\label{sec:appendix-forxarders}

The intuition behind forwarders, quoting \cite{DBLP:conf/ecoop/HondaT91},
is that ``any message can come into the configuration, regardless of the forms of
inner receptors. [\ldots] As the experimenter is not synchronously
interacting with the configuration [\ldots], he may send any message
as he likes.''

In this appendix we give the technical results to ensure that
the function $\liftFWSym$ builds an LTS that satisfies
the axioms of the class \texttt{LtsEq}.


For any LTS $\genlts =\lts{ A }{L}{\st{}}$ of output-buffered agents we assume a function $\outputmultisetSym : A \rightarrow MO$
%defined for any LTS $\genlts_A$ of output-buffered agents such that
\begin{enumerate}[(i)]
\item
  $\co{\aa} \in O(\server)$ if and only if $\co{\aa} \in \outputmultiset{\server}$, and
\item
  for every $\server'$, if $\server \st{\co{\aa}} \server'$ then $\outputmultiset{\server} = \mset{\co{\aa}} \uplus \outputmultiset{\server'}$.
\end{enumerate}
Note that by definition $\outputmultiset{ \server }$ is a finite multiset.


\begin{definition}\label{def:strip-def}
  For any LTS $\genlts = \lts{\States}{L}{\st{}} \in \obaFB$,
  we define the function $\stripSym : \States \longrightarrow \States$ by induction on $
  \outputmultiset{ \server }$ as follows: if
  $\outputmultiset{\server}  = \varnothing $ then
  $\strip{\server} = \server$, while
  if $\exists \co{\aa} \in \outputmultiset{\server}$ and $\server \st{ \co{\aa} } \server'$ then
  $ \strip{ \server } =  \strip{ \server' } $.
  Note that $\strip{ \server }$ is well-defined thanks to the \outputdeterminacy
  and the \outputcommutativity axioms.\hfill$\blacksquare$
\end{definition}


%%   and $$.
%% $$
%% \strip{ \server } =
%% \begin{cases}
%%   \server & \text{ if } \outputmultiset{\server} = \varnothing\\
%%   \strip{ \server' } & \text{ if } \exists  \mu \in \outputmultiset{\server} \text{ and }\server \st{ \mu } \server'
%% \end{cases}
%% $$


%% \begin{lemma}
%%   \label{lem:stout-uniqueness}
%%   Let $\genlts_A \in \oba$ and $\serverA, \serverB_1, \serverB_2 \in A$.
%%   If
%%   $\outnorm{\serverA}{\serverB_1}$ and $\outnorm{\serverA}{\serverB_2}$ then
%%   $\serverB_1 \simeq \serverB_2$.
%% \end{lemma}
%% \begin{proof}
%%   By induction on the cardinality of $\outputmultiset{\serverA}$,
%%   together with an application of the \outputdeterminacy axiom.
%% \end{proof}



We now wish to show that $\liftFW{\genlts} \in \obaFW$ for any LTS
$\genlts$ of output-buffered agents with feedback.
Owing to the structure of our typeclasses, we have first to construct an
equivalence~$\doteq$ over $\liftFW{\genlts}$ that is compatible with the
transition relation, \ie satisfies the axiom in \rfig{Axiom-LtsEq}.
We do this in the obvious manner, \ie by combining the equivalence $\simeq$
over the states of~$\genlts$ with an equivalence over mailboxes.

\begin{definition}
  \label{def:fw-eq}
  For any LTS $\genlts \in \obaFB$, two states $\serverA \triangleright M$
  and $\serverB \triangleright N$ of $\liftFW{\genlts}$ are
  equivalent, denoted
  $\serverA \triangleright M \doteq \serverB \triangleright N$, if
  $ \strip{ \serverA } \simeq \strip{ \serverB }$ and
  $M \uplus \outputmultiset{\serverA} = N \uplus
  \outputmultiset{\serverB}$.\hfill$\blacksquare$
\end{definition}

\begin{lemma}
  \label{lem:harmony-sta}\coqLTS{harmony_a}
  %%%% THIS IS NOT THE HARMONY LEMMA
  For every $\genlts_\StatesA$ and every
  $\serverA \triangleright M, \serverB \triangleright N \in \StatesA \times MO$,
  and every $\alpha \in L$, if
  $
  \serverA \triangleright M \mathrel{({\doteq} \cdot {\sta{\alpha}})}
  \serverB \triangleright N
  $ then
  $
  \serverA  \triangleright M \mathrel{({\sta{\alpha}} \cdot {\doteq})} \serverB' \triangleright N'.
  $
  %%%% OLD STATEMENT %%%
  %% For every $\genlts = \lts{\States}{L}{\st{}}$, %
  %% every $\serverA, \serverB' \in \States$, %
  %% every $M,N' \in MO$ and %
  %% every $\alpha \in L$, if
  %% $
  %% \serverA \triangleright M \mathrel{({\doteq}  \cdot {\sta{\alpha}})} \serverB'
  %% \triangleright N'
  %% $ then %  \text{ implies }
  %% $
  %% \serverA  \triangleright M \mathrel{({\sta{\alpha}} \cdot {\doteq})} \serverB' \triangleright N'.
  %% $
\end{lemma}

%\ilacom{To be more general, in the above paragraph we could replace ``client'' by
%  ``process'' and ``server'' by ``environment''.}

%% \pl{

%% \begin{figure}
%% \hrulefill
%%   $$
%%   \begin{array}{llll}
%%     \begin{prooftree}
%%       \justifies
%%       \server \stout{\varnothing} \server
%%     \end{prooftree}
%%     &
%%     \begin{prooftree}
%%       \server_1 \st{\co{\aa}} \server_2 \qquad
%%       \server_2 \stout{M} \server_3
%%       \justifies
%%       \server_1 \stout{\mset{\aa} \uplus M} \server_3
%%     \end{prooftree}
%%   \end{array}
%%   $$
%%   \caption{The transition $\serverA \stout{M} \serverB$}
%%   \label{fig:rules-stout}
%%   \hrulefill
%% \end{figure}

%% \begin{definition}
%%   Given an LTS $\genlts_{\StatesA}$% $\lts{\States}{L}{\st{}}$
%%   and two states $\serverA, \serverB \in \StatesA$,
%% we define the transition relation $\server \st{M} \serverB$,
%% where $M$ is a multiset of actions, via the following rules
%% \begin{description}
%% \item[\rname{st-m-refl}]
%%   $\server \st{\varnothing} \server$
%% \item[\rname{st-m-mu}]
%%   $\server \st{\mset{\mu} \uplus M} \serverB$ \quad if $\server \st{\mu} \server'$
%%   and $\server' \st{M} \serverB$
%% \end{description}
%% \end{definition}

%% \TODO{QUESTION: should we skip the definition of strip and inline its definition
%%   where it is used ? It would avoid another indirection.}

%% \begin{definition}
%%   Let $\genlts_A \in \oba$.
%%   Given two servers $\serverA$ and $\serverB$ in $A$,
%%   let $\outnorm{\serverA}{\serverB}$ be the relation defined as
%%   $$
%%   \outnorm{\serverA}{\serverB} = \serverA \st{\outputmultiset{\serverA}} \serverB
%%   $$
%% \end{definition}

\begin{lemma}
  \label{lem:fw-eq-id-mb}
  For every $\genlts_\StatesA \in \obaFB$ and every
  $\serverA, \serverB \in \StatesA$, $M \in MO$,
  if $\serverA \simeq \serverB$ then $\serverA \triangleright M \doteq \serverB \triangleright M$.
\end{lemma}

\begin{proof}
This follows from the fact that if $\serverA \simeq \serverB$ then
$\strip{\serverA} \simeq \strip{\serverB}$ and $\outputmultiset{\serverA} = \outputmultiset{\serverB}$.
\end{proof}

%% \begin{figure}
%% \hrulefill
%%   $$
%%   \begin{array}{llllll}
%%     \stproclift &
%%     \begin{prooftree}
%%       \server \st{\alpha} \server'
%%       \justifies
%%       \server \triangleright M \sta{\alpha} \server' \triangleright M
%%     \end{prooftree}
%%     &
%%     \stcommlift &
%%   \begin{prooftree}
%%     \server \st{a} \server'
%%     \justifies
%%     \server \triangleright (\mset{\co{a}} \uplus M) \sta{\tau} \server' \triangleright M
%%   \end{prooftree}
%%   \\[2em]
%%   \stmoutlift &
%%   \begin{prooftree}
%%     \justifies
%%     \server \triangleright (\mset{\co{a}} \uplus M) \sta{\co{a}} \server \triangleright M
%%   \end{prooftree}
%%   &
%%   \stminplift &
%%   \begin{prooftree}
%%     \justifies
%%     \server \triangleright M \sta{a} \server  \triangleright (\mset{\co{a}} \uplus M)
%%   \end{prooftree}
%%   &&
%%   \end{array}
%%   $$
%%   \caption{Lifting of a transition relation to transitions of forwarders.}
%%   \label{fig:rules-liftFW-appendix}
%%   \hrulefill
%% \end{figure}

\noindent
\textbf{\rlem{liftFW-works}}. \textit{For every LTS~$\genlts \in \obaFB$, $\liftFW{\genlts} \in \obaFW$}.

\begin{proof}
  We must show that, given an LTS~$\genlts = \lts{\States}{L}{\st{}} \in \obaFB$, we have that
  $\liftFW{\genlts} \in \obaFW$.
  To do so, we need to show that $\liftFW{\genlts}$ obeys to the axioms given in \req{axioms-forwarders},
  namely \boom and \fwdfeedback.
  We first show that $\liftFW{\genlts}$ obeys to the \boom axiom.

  We pick a process $\serverA \in \States$, a mailbox $M \in \MO$ and a name $\aa \in \Names$.
  The axiom \boom requires us to exhibit a process $\serverA' \in A$ and a mailbox $M' \in \MO$ such that
  the following transitions hold.
  $$
  \serverA \triangleright M \sta{\aa} \serverA' \triangleright M' \sta{\co{\aa}} \serverA \triangleright M
  $$

  We choose $\serverA' = \serverA$ and $M' = (\mset{\co{a}} \uplus M)$.
  An application of the rule \stminplift and then the rule \stmoutlift from \rfig{rules-liftFW}
  allows us to derive the required sequence of transitions as shown below.
  $$
  \serverA \triangleright M \sta{\aa} \serverA \triangleright (\mset{\co{a}} \uplus M) \sta{\co{\aa}} \serverA \triangleright M
  $$

  We now show that $\liftFW{\genlts}$ obeys to the \fwdfeedback axiom.
  To begin we pick three processes $\serverA_1, \serverA_2, \serverA_3 \in \States$, three
  mailboxes $M_1, M_2, M_3 \in \MO$ and a name $\aa \in \Names$ such that:

  $$
  \serverA_1 \triangleright M_1 \sta{\co{\aa}} \serverA_2 \triangleright M_2  \sta{\aa} \serverA_3 \triangleright M_3
  $$

  We need to show that either:
  \begin{enumerate}
  \item $\serverA_1 \triangleright M_1 \sta{\tau} \serverA_3 \triangleright M_3$, or
  \item $\serverA_1 \triangleright M_1 \doteq \serverA_3 \triangleright M_3$
  \end{enumerate}

  We proceed by case analysis on the last rule used to derive the transition
  $\serverA_1 \triangleright M_1 \sta{\co{\aa}} \serverA_2 \triangleright M_2$.
  This transition can either be derived by the rule \stmoutlift or the rule \stproclift .

  We first consider the case where the transition has been derived using the rule
  \stmoutlift. We then have that $\serverA_1 = \serverA_2$ and $M_1 =  (\mset{\co{a}} \uplus M_2)$.
  We continue by case analysis on the last rule used to derive the transition
  $\serverA_2 \triangleright M_2  \sta{\aa} \serverA_3 \triangleright M_3$.
  If this transition has been derived using the rule \stminplift then it must be the case that
  $\serverA_2 = \serverA_3$ and that $(\mset{\co{a}} \uplus M_2) = M_3$.
  This lets us conclude by the following equality to show that
  $\serverA_1 \triangleright M_1 \doteq \serverA_3 \triangleright M_3$.
  $$
  \serverA_1 \triangleright M_1
  = \serverA_2 \triangleright (\mset{\co{a}} \uplus M_2)
  = \serverA_3 \triangleright M_3
  $$
  Otherwise, this transition has been derived using the rule \stproclift, which implies that
  $\serverA_2 \st{\aa} \serverA_3$ together with $M_2 = M_3$.
  An application of the rule \stcommlift ensures the following transition and allows us to
  conclude this case with $\serverA_1 \triangleright M_1 \sta{\tau} \serverA_3 \triangleright M_3$.
  $$
  \serverA_1 \triangleright M_1 = \serverA_2 \triangleright (\mset{\co{a}} \uplus M_2) \st{\tau}
  \serverA_3 \triangleright M_2 = \serverA_3 \triangleright M_3
  $$

  We now consider the case where the transition
  $\serverA_1 \triangleright M_1 \sta{\co{\aa}} \serverA_2 \triangleright M_2$ has been
  derived using the rule \stproclift such that
  $\serverA_1 \st{\co{\aa}} \serverA_2$ and $M_1 = M_2$.

  Again, we continue by case analysis on the last rule used to derive the transition
  $\serverA_2 \triangleright M_2  \sta{\aa} \serverA_3 \triangleright M_3$.
  If this transition has been derived using the rule \stminplift then it must be the case that
  $\serverA_2 = \serverA_3$ and $(\mset{\co{a}} \uplus M_2) = M_3$.
  Also, note that, as $\genlts \in \obaFB$, the transition $\serverA_1 \st{\co{\aa}} \serverA_2$
  implies
  $\outputmultiset{\serverA_1} = \outputmultiset{\serverA_2} \uplus \mset{\co{\aa}}$.
  In order to prove $\serverA_1 \triangleright M_1 \doteq \serverA_3 \triangleright M_3$,
  it suffices to show the following:
  \begin{enumerate}[(a)]
  \item $\strip{ \serverA_1 } \simeq \strip{ \serverA_3 }$, and
  \item $M_1 \uplus \outputmultiset{\serverA_1} = M_3 \uplus \outputmultiset{\serverA_3}$
  \end{enumerate}

  We show that $\strip{ \serverA_1 } \simeq \strip{ \serverA_3 }$ by definition of
  $\strip{}$ together with the transition
  $\serverA_1 \st{\co{\aa}} \serverA_2$ and the equality $\serverA_2 = \serverA_3$.

  The following ensures that
  $M_1 \uplus \outputmultiset{\serverA_1} = M_3 \uplus \outputmultiset{\serverA_3}$.

  $$
  \begin{array}{llll}
  & M_1 \uplus \outputmultiset{\serverA_1}\\
  = & M_1 \uplus \outputmultiset{\serverA_2} \uplus \mset{\co{\aa}} && \text{from } \outputmultiset{\serverA_1} = \outputmultiset{\serverA_2} \uplus \mset{\co{\aa}}\\
  = & M_2 \uplus \outputmultiset{\serverA_3} \uplus \mset{\co{\aa}} && \text{from } M_1 = M_2, \serverA_2 = \serverA_3\\
  = & (M_2 \uplus \mset{\co{\aa}}) \uplus \outputmultiset{\serverA_3}\\
  = & M_3 \uplus \outputmultiset{\serverA_3} && \text{from } M_3 = M_2 \uplus \mset{\co{\aa}}\\
  \end{array}
  $$

  If the transition $\serverA_2 \triangleright M_2  \sta{\aa} \serverA_3 \triangleright M_3$
  has been derived using the rule \stproclift then it must be the case that
  $\serverA_2 \st{\aa} \serverA_3$ and $M_2 = M_3$.
  As $\genlts \in \obaFB$, we are able to call the axiom
  \outputfeedback together with the transitions
  $\serverA_1 \st{\co{\aa}} \serverA_2$ and $\serverA_2 \st{\aa} \serverA_3$
  to obtain a process $\serverA'_3$
  such that $\serverA_1 \st{\tau} \serverA'_3$ and $\serverA'_3 \simeq \serverA_3$.
  An application of \rlem{fw-eq-id-mb} and rule \stproclift allows us to conclude that
  $\serverA_1 \triangleright M_1 \mathrel{({\sta{\tau}} \cdot {\doteq})} \serverA_3 \triangleright M_3$. \qed
\end{proof}
