%%%%COUNTER EXAMPLE COMPLETENESS HENNESSY CASTELLANI 1998
\section{Counter-example to existing completeness result}
\label{sec:counterexample}

%%%%%%%%%%%% NEW MACROS %%%%%%%%%%%%%%%%%%%

\newcommand{\llch}{\ensuremath{\ll_{{\sf ch}}}}
\newcommand{\readyset}{\ensuremath{R}}
\renewcommand\acnvalong{\Downarrow_{a}}

%%%%%%%%%%%%%%%%%%%%%%%%%%%%%%%%%%%%%%%%%% 


In this section we recall the definition of the alternative
preorder $\llch$ by \cite{DBLP:conf/fsttcs/CastellaniH98},
and show that it is not complete with respect to~$\testleqS$,
\ie $\,\testleqS\, \not\subseteq\; \llch$.
  We start with some               
  auxiliary definitions.
  %% \ilacom{add some intuition
  %%   about each definition?
  %% %  (takenfrom~\cite{DBLP:conf/fsttcs/CastellaniH98})
  %%   }
\smallskip

%%%%%%%%% Begin definitions %%%%%%%%%%%%%%%%%%%%%%%

The predicate $\rr{\I}$ is defined by the following two rules: %(\coqMT{guzzler}):
\begin{itemize}
\item $ p \rr{\I} p$ if $p \stable$ and %$\disjoint{I(p)}{I}$,
 $I(p) \cap I = \emptyset$,
  % \ilacom{~~~~Where is the notation $\disjoint{I(p)}{I}$ defined? Why not
  % use $I(p) \cap I = \emptyset$ instead?}
\item $ p \rr{\I \uplus \mset{a}} p'' $ if $\,p \wt{a} p'$ and $ p' \rr{I} p'$
\end{itemize}
\smallskip

The {\em generalised acceptance set} of a process $p$ after a trace
$\trace$ with respect to a multiset of input actions $\I$ is defined by
$$
\gas{p}{s}{\I} = \setof{ O(p'') }{ p \wta{s} p' \rr{\I} p''}
$$

The set of {\em input
  multisets} of a process $p$ after a trace $s$ is defined by
$$\im{p}{s} = \setof{ \mset{\aa_1, \ldots,
    \aa_n } }{ \aa_i \in \Names, p \wta{s} \wt{\aa_1} \ldots
  \wt{\aa_n}}$$

\smallskip

%%%%%%%%%%%%%%%%%%%%%%%%%%%%%%%%%%%%%%%%%%%

The convergence predicate over traces performed by forwarders is
  denoted~$\acnvalong$, and defined as~$\cnvalong$, but over the LTS
  given in \rexa{forwarders-in-TACCS}.

%%%%%%%%%%%%%%%%%%%%%%%%%%%%%%%%%%%%%%%%%%%

The preorder $\llch$ is now defined as follows:

\begin{definition}[Alternative preorder $\llch$~\cite{DBLP:conf/fsttcs/CastellaniH98}]
\label{def:ff-original}
Let $p \llch q$ if for every $s \in \Actfin \wehavethat p \acnvalong s$  implies
\begin{enumerate}
\item $q \acnvalong s$,
\item\label{pt:ff-original-acc-sets}
  for every $\readyset \in \acc{q}{s}$ and every $I \in \im{p}{s}$ such that $I \cap \readyset = \emptyset$ there exists some $O \in \gas{p}{s}{I}$
  such that $O \setminus \co{I} \subseteq \readyset$.\hfill$\blacksquare$
\end{enumerate}
\end{definition}

%% \ilacom{I see you use $p \cnvalong s$ and $q \cnvalong s$ instead of
%%   $p \acnvalong s$ and $q \acnvalong s$ as
%%   in~\cite{DBLP:conf/fsttcs/CastellaniH98}. Is it because in fact the
%%   two predicates $\acnvalong s$ and $\acnvalong s$ coincide?  }

%%%%%%%%%%%%%%%%%%%%%%%%%%%%%%%%%%%%%%%%%%

\smallskip

%%%%%%%%% End definitions %%%%%%%%%%%%%%%%%%%%%%%

\smallskip
%We discuss two technical examples.
%\begin{example}
%\label{ex:rr-1}
We illustrate the three auxiliary definitions using the
  process $\pierre= b.(\tau.\Omega \extc c. \co{d} )$
introduced in ~\rexa{p-testleqS-Nil}. We may infer that
\begin{equation}
  \label{eq:pierre-rr}
  \pierre \rr{\mset{b,c}} \mailbox{\co{d}}
\end{equation}
thanks to the following derivation tree
$$
\begin{prooftree}
  \begin{prooftree}
    \begin{prooftree}
      \justifies
      \out{d} \rr{\emptyset}  \out{d}
      \using
      \out{d} \stable \text{ and } I(\out{d}) \cap \emptyset = \emptyset
    \end{prooftree}
    \justifies
    \tau.\Omega \extc c.\co{d} \rr{\mset{c}} \out{d}
    \using
    \tau.\Omega \extc c.\co{d} \wt{c} \out{d}
  \end{prooftree}
  \justifies
  \pierre \rr{\mset{b,c}} \out{d}
  \using
  \pierre \wt{b} \tau.\Omega \extc c.\co{d}
\end{prooftree}
$$



%\begin{example}
%%  \label{ex:rrI-divergence}\label{ex:gas-p}
Let us now consider the generalised acceptance set of \pierre after the
trace $\varepsilon$ with respect to the multiset $\mset{b, c}$.
We prove that
\begin{equation}
  \label{eq:pierre-gas}
  \gas{ \pierre }{ \varepsilon }{ \mset{b, c} } = \set{ \set{ \co{d} } }
\end{equation}

By definition 
  $\gas{ \pierre }{ \varepsilon }{ \mset{b, c} } =  \setof{ O(p'') }{
  \pierre \wta{ \varepsilon} p' \rr{\mset{b, c}} p'' }$. Since $
\pierre \stable$, we have
  \begin{equation}
    \label{eq:gas-p}
    \gas{ \pierre }{ \varepsilon }{ \mset{b, c} } =  \setof{ O(p'') }{ \pierre \rr{\mset{b, c}} p'' }
  \end{equation}
  Then, thanks to \req{pierre-rr} %that $ \pierre \rr{\mset{b,c}} \out{d} $,
  we get
  $
  O(\co{d}) = \set{ \co{d} } \in \gas{ \pierre }{ \varepsilon }{ \mset{b, c} }
  $.
  We show now that  $\set{ \co{d} }$ is the only element of this
  acceptance set.
  %, namely that the following set inclusion, $\gas{ \pierre }{ \varepsilon }{ \mset{b, c} } \%subseteq \set{\set{ \co{d} }}$.
  % Pick an $O \in \gas{p}{ \varepsilon }{ \mset{b, c} }$.
  % Thanks to (\ref{eq:gas-p}) above there exists a $p''$ such that
  % $\pierre \rr{\mset{b, c}} p''$ and that $O(p'') = O$.
By (\ref{eq:gas-p}) above, it is enough to show that
$\pierre \rr{\mset{b, c}} p''$ implies $p'' = \out{d} $.
Observe
that %This is the case because
  \begin{enumerate}
  \item $ I( \pierre ) \cap \mset{ b,c } \neq \emptyset $,
  \item $ \pierre \wt{ \aa } p'$ implies $\aa = b$, and
  \item
    There are two different states $p'$ such that
   $ \pierre\wt{b}p'$, but 
    the only one that can do the input $c$ is $p' = \tau.\Omega \extc c.\co{d}$.
  \end{enumerate}
  This implies that the only way to infer
  $\pierre \rr{\mset{b, c}} p''$ is via the derivation tree
  that proves \req{pierre-rr} above. Thus~$p'' = \co{d} $.

% The last notation we need is $\im{p}{s} = \setof{ \mset{\mu_1, \ldots,
%     \mu_n } }{ \mu_i \in \Names, p \wta{s} \wt{\mu_1} \ldots
%   \wt{\mu_n}}$. Essentially $\im{p}{s}$ is the set of {\em input
%   multisets} of $p$ after $s$.

% \begin{definition}%[\cite{DBLP:conf/fsttcs/CastellaniH98}]
% \label{def:ff-original}
% Let $p \leq q$ if for every $s \in \Actfin \wehavethat p \acnvalong s$  implies
% \begin{enumerate}
% \item $q \acnvalong s$,
% \item\label{pt:ff-original-acc-sets}
%   for every $A \in \acc{q}{s}$ and every $I \in \im{p}{s}$ such that $I \cap A = \emptyset$ there exists some $O \in \gas{p}{s}{I}$
%   such that $O \setminus \co{I} \subseteq A$.\hfill$\blacksquare$
% \end{enumerate}
% \end{definition}

\begin{counterexample}
  \label{ex:ffCH98-incomplete}
  The alternative preorder $\llch$
  is not complete for $\testleqS$, namely $p \testleqS q$ does not imply $p \llch q$.
\begin{proof}
  The cornestone %(``pierre angulaire'' ;-))
  of the proof is the process
 $\pierre = b.(\tau.\Omega \extc c. \co{d} )$ discussed above.
In \rexa{p-testleqS-Nil} we have shown that $\pierre \testleqS \Nil$.
Here we show that $\pierre \not\llch \Nil$, because the pair $(\pierre,\Nil)$ does not
satisfy Condition \ref{pt:ff-original-acc-sets} of \rdef{ff-original}.

Since $\pierre \stable$, we know that $\pierre \conv$, and thus by
definition $\pierre \acnvalong \varepsilon$. We also
have by definition $\acc{\Nil}{\varepsilon} = \set{ \emptyset }$, and
$\im{\pierre}{\varepsilon} = \set{\emptyset, \mset{ b}, \mset{ b, c
  }}$.

%% \ilacom{\TODO{Please check this. Previously it was:
%%   $\im{p}{\varepsilon} = \mset{ b, c }$, which was wrong. Indeed,
%%   $\im{p}{s} $ is a set of multisets, not a single multiset. Moreover,
%%   if $p$ can do a transition sequence
%%   $ p \wta{s} p_0 \wt{\aa_1} p_1 \ldots \wt{\aa_n} p_n$, then
%% % $$\im{p}{s} = \setof{ \mset{\aa_1, \ldots,
%% %     \aa_n } }{ \aa_i \in \Names, p \wta{s} \wt{\aa_1} \ldots
%% %   \wt{\aa_n}}$$
%%   it can also do any prefix of it (since there is no requirement on
%%   the state $p_n$). Therefore, $\im{p}{s}$ always contains
%%   $\emptyset$, as well as the multiset $\mset{\aa_1, \ldots, \aa_i }$
%%   for any $i$ such that $1\leq i\leq n$.  This mistake seems to have
%%   no consequences here, since it is enough to find a single $I$ that
%%   violates Condition \ref{pt:ff-original-acc-sets} and this $I$ is
%%   $\mset{ b, c }$.}}

Let us check Condition \ref{pt:ff-original-acc-sets} of
  \rdef{ff-original} for $p = \pierre$ and $q = \Nil$. Since there is
  a unique $\readyset\in \acc{\Nil}{\varepsilon}$, which is
  $\emptyset$, and $I \cap \emptyset = \emptyset$ for any $I$, we only
  have to check that for every $I \in \im{\pierre}{\varepsilon}$ there
  exists some $O \in \gas{\pierre}{\varepsilon}{I}$ such that
  $O \setminus \co{I} \subseteq \emptyset$.

Let $I = \mset{ b, c }$.
%, clearly $I \cap \emptyset = \emptyset$.
%It suffices to show that for every $ O \in  \gas{\pierre}{ \varepsilon }{ I }$
%we have $O \setminus \co{I} \not\subseteq \emptyset$.
By~\req{pierre-gas} it must be $O =\set{ \co{d} }$. Since
  $ \set{ \co{d} } \setminus \co{I} = \set{ \co{d} } \setminus \co{
    \mset{ b, c } } = \set{ \co{d} } \not\subseteq \emptyset$, the
  condition is not satisfied. Thus $\pierre \not\llch \Nil$.
\end{proof}
\end{counterexample}


%% The condition proposed in \cite[Definition
%%   5]{DBLP:conf/fsttcs/CastellaniH98} to define the alterative preorder is similar to 
%% \begin{equation}
%%   \label{eq:incomplete-condition}
%% \forevery A \in \acc{q}{s} \and \I \in \MI \suchthat
%% \intersection{\I}{A} = \emptyset \text{ there exists some } O \in \gas{p}{s}{I}
%%   \suchthat O \setminus \co{I} \subseteq A.
%% \end{equation}
%% In the next example we show that this property is not complete.

%% \begin{counterexample}
%%   \label{ex:ffCH98-incomplete-1}
%%   \pierre is again at the center of the action.
%%   In \rexa{p-testleqS-Nil} we have proven \req{running-example-1}, namely that 
%%   $\pierre \testleqS \Nil$.
%%   In this example we show that $ p \not \bhvleq \Nil$,
%%   because the pair $(\pierre, \Nil)$ does not respect condition
%%   \ref{pt:ff-original-acc-sets} of \req{incomplete-condition}.
  
%%   Since $ \pierre \stable$, we know $\pierre \conv$, and thus by definition we 
%% have $\pierre \acnvalong \varepsilon$. By definition we also know
%% that $\acc{\Nil}{\varepsilon} = \set{ \emptyset }$,
%% and that $\mset{b,c} \in \MI$. %%$\im{p}{\varepsilon} = \mset{ b, c }$.

%% Let $I = \mset{ b, c}$, clearly $\I \cap \emptyset = \emptyset$.
%% It suffices to show that for every $ O \in  \gas{ \pierre }{ \varepsilon }{ \I }$
%% we have $O \setminus \co{\I} \not\subseteq \emptyset$.
%% We have proven in \rexa{gas-p}
%% that $ \gas{ \pierre }{ \varepsilon }{ \I } = \set{\set{ \co{d} } }$,
%% and the following euqlities hold $ \set{ \co{d} } \setminus \co{\I} = \set{ \co{d} } \not\subseteq \emptyset$.\qed
%% \end{counterexampl}
