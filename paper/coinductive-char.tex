%%%%COUNTER EXAMPLE COMPLETENESS HENNESSY CASTELLANI 1998
\section{Co-inductive characterisation of the \mustpreorder}
\label{sec:coinductive-char}


\begin{definition}[Co-inductive characterisation of the \mustpreorder]
\label{def:coinductive-char}
For every finite LTS $\genlts_\StatesA$, $\genlts_\StatesB$ and every
$X \in \pparts{\StatesA}, \serverB \in \StatesB$,
the coinductive characterisation $\coindleq$ of the \mustpreorder is defined as the greatest relation
such that whenever $X \coindleq \stateB$, the following requirements hold:
\begin{enumerate}
\item $X \downarrow \text{ implies } \serverB \downarrow$,
\item\label{pt:coind-tau-serverB} For each $\serverB'$ such that $\serverB \st{\tau} \serverB'$, we have that $X \coindleq \serverB'$,

\item\label{pt:coind-acceptance-sets}
  $X \downarrow$ and $\serverB \stable$ imply that there exist
  $\state \in X$ and $\stateA \in \StatesA$ such that
  $\state \wt{} \stateA \stable$ and $O(\stateA) \subseteq O(\stateB)$,

\item\label{pt:coind-continuations-mu} For any $\mu \in \Act$,
  if $X \cnvalong{\mu}$,
  then for every  $X'$ and $\stateB'$
  such that $X \wt{\mu} X'$ and $\stateB \st{\mu} \stateB'$,
%  $X' \in  \pparts{\StatesA}$ and $\stateB \in \StatesB$
 % such that 
  we have that
  $X' \coindleq \serverB'$.
\end{enumerate}
\end{definition}


\begin{theorem}
\label{thm:coinductive-char-equiv}
For every LTS $\genlts_\StatesA$, $\genlts_\StatesB \in \oba$, every
$\state \in \StatesA$ and $\serverB \in \StatesB$, we have that
$\state \testleqS \serverB$ if and only if $FW(\set{\state}) \coindleq FW(\serverB)$ .
\end{theorem}

\newcommand{\Rfw}{\mathrel{\R_{\textsf{fw}}}}
\newcommand{\Rbase}{\mathrel{\R_{\textsf{base}}}}

\gb{
  We can now prove \req{mailbox-hoisting}. To this end
  recall that $p = \tau.(\co{\aa} \Par \co{b}) \extc \tau.(\co{\aa} \Par \co{c})$
  and $q = \co{\aa} \Par (\tau.\co{b} \extc \tau.\co{c})$.
  First we gather together the pairs due to the transitions on the standard LTS:
  let
  $$
  \begin{array}{lll}
    \Rbase & = &
    \set{ (\set{ p }, q), (\set{ p }, \co{\aa} \Par \co{b}), (\set{ p },  \co{\aa} \Par \co{c}) }
    \\
    & \cup & \set{ (\set{ \co{b}, \co{c} },  \tau.\co{b} \extc \tau.\co{c}) }\\
    & \cup & \set{ (\set{  \co{b}, \co{c} }, \co{b}),   (\set{ \co{b}, \co{c} }, \co{c})}\\
    & \cup & \set{ (\set{ \co{\aa} },  \co{\aa}) }
    \\
    & \cup & \set{ (\set{ \Nil }, \Nil) }
  \end{array}
  $$
  and let
  $$
  \liftFW{ \R } = \setof{ (\liftFW{ X }, \liftFW{ \stateB }) }{ X \R q }
  $$
  The relation $\liftFW{\Rbase}$ does not satisfy the fourth condition on continuations,
  because for instance
  $$
  (\set{ \Nil } \triangleright \varnothing) \wt{ \aa } \set{ \Nil }  \triangleright \mset{ \co{\aa} },
  \qquad
  (\Nil \triangleright \varnothing) \st{ \aa } \Nil  \triangleright \mset{ \co{\aa} }.
  $$
  and $(\set{ \Nil }  \triangleright \mset{ \co{\aa} },  \Nil  \triangleright \mset{ \co{\aa} }) \not \in {\liftFW{\Rbase}}$.
  The first intuition thus is to close binary relations wrt mailboxes via the following function:
  $$
  \begin{array}{lll}
    f_1(\R) & = &\setof{ (X \Par \mailbox{ \Pi M },  q \Par \mailbox{ \Pi M })  }{ X \R q }
%    f_2(\R) & = & \setof{ (X ,  q \Par \mailbox{ \Pi M })  }{ X \R q } \\
  \end{array}
  $$

  Observe that\footnote{If the equivalence via $\doteq$ is not true, then we have use to use strong bisimilarity $\sim$.}
  $$
  \liftFW{ \server \Par \mailbox{ \Pi M } } =
  (\server \Par \mailbox{ \Pi M }) \triangleright \varnothing
  \doteq
  \server \triangleright M
  $$

  We are now ready to unfold the proof.

  \begin{lemma}
    $ \serverA \testleqS \serverB$.
  \end{lemma}
  \begin{proof}
    Let $\hat{\R} = {\liftFW{\Rbase}} \cup f_1(\liftFW{\Rbase})$.
    By construction
    $$
    \liftFW{ \set{ \serverA } } \mathrel{\hat{\R}} \liftFW{ \serverB },
    $$
    and thus \rthm{coinductive-char-equiv} ensures that it suffices to show that
    the relation~$\hat{\R}$ satisfies the conditions in \rdef{coinductive-char}.
    
    Since~$\Rbase$ satisfies \rdefpt{coinductive-char}{coind-tau-serverB},
    WHY DOES $\hat{\R}$ satify the property?

    Since~$\Rbase$ satisfies \rdefpt{coinductive-char}{coind-acceptance-sets},
    \rlem{liftFW-preserves-coind-acceptance-sets} and \rlem{f1-preserves-coind-acceptance-sets}
    imply that~$\hat{\R}$ satisfies \rdefpt{coinductive-char}{coind-acceptance-sets}.

    Since~$\Rbase$ satisfies \rdefpt{coinductive-char}{coind-continuations-mu}
    \rlem{preservation-coind-continuations-mu-1} and \rlem{preservation-coind-continuations-mu-2}
    imply that also~$\hat{\R}$ satisfies that property.
  \end{proof}


  \begin{lemma}
    \label{lem:liftFW-preserves-coind-acceptance-sets}
    Suppose a relation $\R$ satisfies \rdefpt{coinductive-char}{coind-acceptance-sets},
    and for every $(Y,q) \in {\R}$ the processes in $Y$ does not perform input actions.
    Then \rdefpt{coinductive-char}{coind-acceptance-sets} is satisfied also by the relation $\liftFW{\R}$.
  \end{lemma}

  
  \begin{lemma}
    \label{lem:f1-preserves-coind-acceptance-sets}
    Suppose a relation $\R$ satisfies \rdefpt{coinductive-char}{coind-acceptance-sets},
    and for every $(Y,q) \in {\R}$ the processes in $Y$ does not perform input actions.
    Then \rdefpt{coinductive-char}{coind-acceptance-sets} is satisfied also by the relation $f_1(\R)$.
  \end{lemma}
  \begin{proof}
    Fix a pair $(X, q) \in f_1(\R)$ and suppose that $q \stable$ and $X \conv$.
    We have to show that for some $ O \in \acc{X}{\varepsilon}$ such that $O \subseteq O(q)$.
    By construction we know that for some multiset of output actions $M$, set $Y$,
    and process $q'$ we have that 
    $(Y, q') \in {\R}$, $q = q' \Par \mailbox{\Pi M}$ and
    $X = \setof{ (p \Par \mailbox{\Pi M}) }{ p \in Y } $.

    We know by hypothesis that there exists a set $\hat{O} \in \acc{Y}{\varepsilon}$
    such that $\hat{O} \subseteq O(q')$.
    We thus obtain $ \hat{O}\cup M \subseteq O(q') \cup M$,
    and thus $\hat{O} \cup M \subseteq O(q)$.
    Since $\hat{O} \cup M \in \acc{X}{\varepsilon}$ we can conclude.\qed
  \end{proof}


  %% \begin{lemma}
  %%   Suppose a relation $\R$ satisfies \rdefpt{coinductive-char}{coind-acceptance-sets},
  %%   and for every $(Y,q) \in \R$ the processes in $Y$ performs only output actions.
  %%   Then the condition is satisfied also by the relation $f_2(\R)$.
  %% \end{lemma}
  %% \begin{proof}
  %%   Fix a pair $(X, q) \in f_2(\R)$ and suppose that $q \stable$ and $X \conv$.
  %%   We have to show that for some $ O \in \Acc{X}{\varepsilon}$ we have $O \subseteq O(q)$.
  %%   By construction we know that for some multiset of output actions $M$,
  %%   and process $q'$ we have that $(X, q') \in \R$, $q = q' \Par \mailbox{\Pi M}$.

  %%   We know by hypothesis that there exists a set $\hat{O} \in \Acc{X}{\varepsilon}$
  %%   such that $\hat{O} \subseteq O(q')$.
  %%   We thus obtain $ \hat{O} \subseteq O(q') \cup M$,
  %%   and thus $\hat{O} \subseteq O(q)$.
  %%   Since $\hat{O} \in \acc{X}{\varepsilon}$ we can conclude.\qed
  %% \end{proof}


    
  \begin{lemma}
    For any $\R$ that satisfies \rdefpt{coinductive-char}{coind-continuations-mu},
    then also the relation $\liftFW{\R} \cup f_1(\liftFW{\R})$ satisfies \rdefpt{coinductive-char}{coind-continuations-mu}.
  \end{lemma}

  


  
  \begin{lemma}
    \label{lem:preservation-coind-continuations-mu-1}
    For any $(X, \serverB) \in {\Rbase}$,
    and any $\mu \in \Act$,
    such that $\liftFW{X} \cnvalong{\mu}$, $\liftFW{X} \wt{\mu} X'$ and $\liftFW{\stateB} \st{\mu} \stateB'$,
    we have that $(X', \serverB') \in \liftFW{\Rbase } \cup f_1(\liftFW{\Rbase} )$.
  \end{lemma}
  \begin{proof}
    Fix a pair $(X, q) \in {\Rbase}$, %
    a visible action $\mu$ such that $X \cnvalong{\mu}$
    and $\liftFW{ X } \wt{\mu} X'$ and $\liftFW{\stateB} \st{\mu} \stateB'$
    for some $ X'$ and $\serverB'$.
    
    We know by definition of $\liftFW{-}$ that 
    $$
    \begin{array}{lll}
      \liftFW{ X } & = & X \triangleright \varnothing,\\
      \liftFW{ \stateB } & = & \stateB \triangleright \varnothing.
    \end{array}
    $$

    
    We proceed by case analysis on the rule used to derive the transition
    $$
    \liftFW{\stateB} \st{\mu} \stateB'.
    $$
    According to \rfig{rules-liftFW-appendix}
    there are four cases, but \rname{L-Comm} cannot have been used,
    because~$\mu$ is a visible action, and \rname{L-Mout} cannot have been used either,
    because in~$\liftFW{\stateB}$ the mailbox is empty.


    If rule \rname{L-Proc} was used, then the transition is due to the standard LTS,
    so for some $\hat{\stateB}$ we have that
    $$
    \begin{array}{l}
      \stateB' = \hat{\stateB} \triangleright \varnothing,\\
      \stateB \st{\mu} \hat{\stateB}
    \end{array}
    $$
    moreover the construction of $\Rbase$ ensures that $\mu$ is an output action.
    
    We now reason on the weak transition $ \liftFW{ X } \wt{ \mu } X'$.
    Wi discuss first the special case in which the transition is actually strong, i.e. there are no tau steps.
    The transition at hand must be due to either \rname{L-Proc} or \rname{L-Mout}, but
    in $ \liftFW{ X } $ the mailbox is empty, hence the transition cannot
    be due to \rname{L-Mout}. This means that the transition is due to the standard LTS, so for
    some $\hat{X}$ we have
    $$
    \begin{array}{l}
      X' = \hat{X} \triangleright \varnothing,\\
      \hat{ X } \st{\mu} \hat{X}.
    \end{array}
    $$
    But then $( \hat{X}, \hat{\stateB}) \in \Rbase$, because by hypothesis $\R$ satisfies
    \rdefpt{coinductive-char}{coind-continuations-mu}.
    It follows that $(X', q') \in \liftFW{\Rbase}$ 

    TODO: discuss the case in which there are taus before or after the action $\mu$.

    
    If the rule \rname{L-Minp} was used, then
    $$
    \stateB' = q \triangleright \mset{\co{\mu}},
    $$
    the action $\mu$ is an input.
    The construction of $\Rbase$ ensures that $X$ performs only output actions,
    hence the transition $\liftFW{ X } \st{\mu} X' $ must be due to
    the rule  \rname{L-Minp}, and thus
    $$
    X' = X \triangleright \mset{\co{\mu}}
    $$
    It follows that $(X', q') \in f_1(\liftFW{\Rbase})$.

    TODO: discuss the case in which there are taus before or after the action $\mu$. \qed
  \end{proof}

  

  
  \begin{lemma}
    \label{lem:preservation-coind-continuations-mu-2}
    For any $(X, \serverB) \in f_1( \Rbase )$,
    and any $\mu \in \Act$,
    such that $\liftFW{X \cnvalong{\mu}}$, $\liftFW{X} \wt{\mu} X'$ and $\liftFW{\stateB} \st{\mu} \stateB'$,
    we have that $(X', \serverB') \in \liftFW{\Rbase} \cup f_1(\liftFW{\Rbase} )$.
  \end{lemma}
  \begin{proof}
    Fix a pair $(X, q) \in f_1(\Rbase)$ and a visible action $\mu$ such that $X \cnvalong{\mu}$
    and $\liftFW{ X } \wt{\mu } X' $ and $\liftFW{\stateB} \st{\mu} \stateB' $.

    By construction for some $M$ we have
    $$
    \begin{array}{lll}
      \liftFW{ X } & = & (X \Par \mailbox{\Pi M}) \triangleright \varnothing,\\
      \liftFW{ \stateB } & = & (\stateB \Par \mailbox{\Pi M}) \triangleright \varnothing.
    \end{array}
    $$

    We proceed by case analysis on the rule used to derive the transition
    $\liftFW{\stateB} \st{\mu} \stateB' $.
    According to \rfig{rules-liftFW-appendix}
    there are four cases, but \rname{L-Comm} cannot have been used,
    because~$\mu$ is a visible action, and \rname{L-Mout} cannot have been used either,
    because in~$\liftFW{\stateB}$ the mailbox is empty.

    Case \rname{L-Proc}
    
    TODO: FINISH ME

    Case \rname{L-Minp}

    TODO: FINISH ME

    
    \bigskip
    
    OLD PROOF. AVOID IT.
    
    Either the transition of $X$ is due to forwarding or to the standard LTS.
    If the transition is due to forwarding then $X' = \hat{X} \Par \mailbox{\Pi M \Par \co{ \mu} } $
    and $\mu$ is an input action. Since in the original LTS no state can perform any input,
    the transition $\liftFW{\stateB}  \st{\mu} \liftFW{\stateB'} $ must also be due to forwarding,
    thus $q' =  \hat{q} \Par \mailbox{\Pi M \Par \co{ \mu}}$,
    and thus $(X', q') \in f_1(\Rfw)$.

    
    If the transition of $X$ is due to the standard LTS then
    $$
    \hat{X} \Par \mailbox{\Pi M} \wt{\mu} \hat{X}' \Par \mailbox{\Pi M'}
    $$
    and $\mu$ is an output action.
    Either $\hat{X} = \hat{X}'$ or $\mailbox{\Pi M} = \mailbox{\Pi M'}$.

    In the first case  $M' = M \setminus \mset{ \mu }$, and
    $(\hat{X}, \hat{q}) \in \Rfw$.
    We have to find some mailbox $\hat{M}$ such that
    $$
    \begin{array}{lll}
      X' & = & \hat{X}' \cup \Par \mailbox{\Pi \hat{M}}\\
      q' & = & \hat{q}' \cup \Par \mailbox{\Pi \hat{M}}\\
    \end{array}
    $$
    with $( \hat{X}',  \hat{q}') \in \Rfw$.

    If the transition in $q$ is due to the mailbox then
    $q' = \hat{q}$ and thus we have $(\hat{X}' \Par \mailbox{\Pi M'}, \hat{q} \Par \mailbox{\Pi M'}) \in f_1(\Rfw)$.
    If the transition is due to $\hat{q}$ then
    $ q' = \hat{q}' \Par \mailbox{\Pi M}$, and
    output-determinacy implies
    that
    $$
    q' =  \hat{q}' \Par \mailbox{\Pi M} \simeq \hat{q} \Par \mailbox{\Pi M \setminus \mset{ \mu }} = \hat{q} \Par \mailbox{\Pi M'}
    $$
    It now follows that $ (X', q') \in f_1(\Rfw) $.\footnote{Technically, we need to close the relations wrt $\simeq$.}
    \qed
  \end{proof}
  
  }
